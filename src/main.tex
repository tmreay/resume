% resume.tex

% Document settings -----------------------------------------------------------

\documentclass{article}
\usepackage[a4paper, margin=2cm]{geometry}
\usepackage[T1]{fontenc}
\usepackage{enumitem}

\pagestyle{empty}
\setlist[1]{itemsep=0em}
\raggedbottom
\raggedright

% Custom commands -------------------------------------------------------------

\newcommand{\heading}[1]{
	\vspace{2em}
	\textbf{\large #1} \\
	\line(1, 0){\textwidth}
	\vspace{-0.5em}
}
\newcommand{\subheading}[2]{
	\begin{tabular*}{\textwidth}{l@{\extracolsep{\fill}}r}
		\textbf{#1} & #2 \\
	\end{tabular*} \\
}
\newcommand{\itemheading}[2]{
	\vspace{0.2em}
	\begin{tabular*}{\textwidth}{l@{\extracolsep{\fill}}r}
		\textit{#1} & \textit{#2} \\
	\end{tabular*} \\
	\vspace{-0.5em}
}

% Resume ----------------------------------------------------------------------

\begin{document}

\textbf{\huge Thomas Matthew Reay}

\vspace{1em}

\begin{tabular*}{\textwidth}{l|l|l|l}
	\hline
	Tokyo, JP & 080-5000-2399 & tmattreay@gmail.com & thomasreay.info
	\rule{0em}{1.25em} \\
\end{tabular*}

\vspace{1em}

Innovative and results-oriented professional offering six years of experience
in full-stack web development, desktop application development, data analysis,
consulting, and client success, additionally leveraging a Bachelor of Science
with a concentration in Computational Physics, to realize exemplary results and
desired outcomes.

\heading{Work Experience}

\subheading{Enthought}{Austin, TX | Tokyo, JP}
\itemheading{Scientific Software Developer}{March 2018 - Aug. 2020}
\vspace{0.5em}
\itemheading{Senior Scientific Software Developer}{Aug. 2020 - Present}
\begin{itemize}
	\item Collaborated with clients in the semi-conductor industry to develop
	      applications and technical solutions for domain-specific problems.
	\item Designed a web application for text and image search with 1,000's of
	      users, using React, Flask, Elasticsearch, Keras, and PostgreSQL.
	\item Designed a content extraction pipeline for processing Office and PDF
	      documents, using Python, Java and the Libreoffice UNO API.
	\item Managed multiple deployments of a distributed web application in an
	      Azure-hosted environment, using Ansible, Docker, and NGINX.
	\item Maintained a code base of ~130K LOC for 4+ years, and taught that
	      code base to the client's development team for future maintenence.
	\item Designed an SEM image processing desktop application with 100's of
	      users, using Python, Keras, and Qt.
	\item Assisted with mentoring 5 separate cohorts of client engineers, as
		  they learned Python and developed software to provide business value
		  to their teams.
	\item Assisted with onboarding 14 developers to different aspects of the
	      client engagement, and served as a mentor to 3 new hires.
\end{itemize}
\itemheading{Tokyo Office Manager}{June 2020 - Present}
\begin{itemize}
	\item Provided general management oversight for an office with 7
		  employees.
	\item Coordinated two office transfers to accommodate for increased
		  facility needs and growing staff size.
\end{itemize}

\subheading{Applied Research Laboratories: Space and Geophysics Lab}{Austin, TX}
\itemheading{Research Engineering / Scientist Associate}{Sept. 2015 - March 2018}
\begin{itemize}
	\item Built a software suite in Python for beam-forming and array signal
	      processing.
	\item Produced visualizations of various geo-spatial data for academic
	      conferences and publications.
	\item Assisted in the setup and operation of field experiments and data
	      collection in challenging environments.
	\item Served as point of contact for testing and verification of software
	      from three partner labs.
\end{itemize}
\itemheading{Student Technician}{Nov. 2014 - May 2015}
\begin{itemize}
	\item Analyzed ionospheric data from the GNSS satellite network.
\end{itemize}

\subheading{The University of Texas at Austin: Raizen Group}{Austin, TX}
\itemheading{Research Engineering / Scientist Associate}{June 2015 - Aug. 2015}
\begin{itemize}
	\item Researched potential applications for graphene oxide structures.
\end{itemize}
\itemheading{Research Assistant}{June 2014 - Oct. 2014}
\begin{itemize}
	\item Designed an atomic hydrogen source for use in an ultra high-vacuum
	      environment.
\end{itemize}

\heading{Education}

\subheading{The University of Texas at Austin}{Austin, TX}
\itemheading{BS, Computational Physics (GPA: 3.88; Hours: 137)}{Graduated with Honors, Spring 2015}

\heading{Core Competencies}

Application Development, Application Deployment, Domain-driven Design, Troubleshooting,
Requirements Modeling, Process Enhancement, Optimization Techniques,
Workflow Orchestration, Cross-functional Collaboration, Issue Resolution,
Personnel Management, Technical Mentorship

\heading{Technical Skills}

\textbf{Programing Languages:}
\begin{itemize}
	\item[] Python, Javascript, Typescript, SQL, Java, Fortran, C, \LaTeX,
	        Mathematica, MatLab
\end{itemize}

\textbf{Tools and Frameworks:}
\begin{itemize}
	\item[] React, Flask, Docker, Ansible, PostgreSQL, Elasticsearch,
	        Libreoffice, Keras, Azure
\end{itemize}

\end{document}
